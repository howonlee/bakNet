\documentclass{beamer}
\usepackage{amsmath}
\usepackage{amssymb}
\usepackage{graphicx}
\begin{document}
\title{Extremal Optimization}
\author{Howon Lee}
\maketitle

\begin{frame}
  \frametitle{What is the question?}
  How can you use extremal dynamics to train some kind of neural net?
\end{frame}

\begin{frame}
  \frametitle{What? What's that?}
  Qualitative model of evolution: Bak-Sneppen model.
  \begin{figure}
    \includegraphics{bak_sneppen}
  \end{figure}

  Rules of Bak-Sneppen model. How does it behave? Criticality. Self-organized criticality.

  Caveats: CR Shalizi, W Tozier, "A Simple Model of the Evolution of Simple Models of Evolution"
\end{frame}

\begin{frame}
  \frametitle{Approaches Taken}
  Bak and Chialvo's model

  More abstract: extremal optimization

  Finally: Using $\tau$-EO on RBM, classification results
\end{frame}

\begin{frame}
  \frametitle{Adaptive Learning by Extremal Dynamics and Negative Feedback}
  How it works, from 20000 feet
  \begin{figure}
    \includegraphics{bak_chialvo_net_topology}
  \end{figure}

  Problems: Conjunctive neurons
  \begin{figure}
    \includegraphics{bak_plot}
  \end{figure}
  
  Comparison to simulated annealing
\end{frame}

\begin{frame}
  \frametitle{More abstractly: EO, $\tau$-EO}
  How it works, from 20000 feet

  Existing results, from Boetticher: TSP, Ising model

  $\tau$-EO
  \begin{figure}
    \includegraphics{eo_alg}
  \end{figure}

  The hope:
  \begin{figure}
    \includegraphics{boettcher}
  \end{figure}
\end{frame}

\begin{frame}
  \frametitle{$\tau$-EO on RBM}
  Why not feedforward? BM vs. Ising Model

  Locality: RBM

  \begin{figure}
    \includegraphics{rbm_eq}
  \end{figure}

  Ising model learning dynamics
  \begin{figure}
    \includegraphics{2000}
  \end{figure}
  \begin{figure}
    \includegraphics{10000}
  \end{figure}
  \begin{figure}
    \includegraphics{ising_model_unzoomed}
  \end{figure}
  \begin{figure}
    \includegraphics{ising_model_zoomed}
  \end{figure}
\end{frame}

\begin{frame}
  \frametitle{Task}
  See performance in RBM learning, as simple as possible

  Bit string: metaphor of Hamming cliff in GA

  Actually, this is just a weird coordinate descent
\end{frame}

\begin{frame}
  \frametitle{Results on RBM}
  \begin{figure}
    \includegraphics{eo_rbm_unzoomed}
  \end{figure}
  \begin{figure}
    \includegraphics{eo_rbm_zoomed}
  \end{figure}
\end{frame}

\begin{frame}
  \frametitle{Problems, Issues}
  Make an algorithm that actually differs from CDiv
  Try non-restricted RBM, and learn with regular gradient
  (using EO in place of SA)
\end{frame}

\begin{frame}
  End

  %%% citations
\end{frame}

\end{document}

