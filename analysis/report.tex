\documentclass[12pt]{article}
\usepackage{amsmath}
\usepackage{amssymb}
\usepackage{graphicx}
\usepackage[margin=1in]{geometry}
\newcommand{\del}{\nabla}
\begin{document}

\title{Psych209 Final Project}
\author{Howon Lee}
\maketitle

%%%% to talk about

%explain the bak net
%claim of bak and chialvo
%evolutionary background
%performance replications
%dismal actual performance on parity bit task
%try it on digits
%try it as a clusterer
%think about how it goes right if it goes right

%%%% organization

\section{Stuff} %%%%%

\section{} %What is the issue or question you will be addressing?

\section{} %Why is it interesting, what has previously been done, and what remains to be done?

\section{Novel Approach} %What approach will you be taking to address it? (at a big picture level; novel and/or otherwise noteworthy aspects)

\section{Details of Approach}%Specific target phenomenon or phenomena you will be addressing: e.g., pattern of data you intend to try to fit. Model network task setting, architecture, processing and learning algorithm, training environment (corpus used for training, knowledge base built in to network),

\section{Results and Analysis}

%%First present your primary findings that bear directly on the target phenomena.

%%A strong paper will, in addition, present an analysis of why the results came out the way they did, especially in cases where results did not come out as expected. 

%%It can be useful to discuss results you obtain with one of us to get suggestions as to how to fully understand your findings.

%%Analysis not only of network outputs but also of the structure of the information present in the materials you use to train your network (when relevant) and / or the hidden unit activations, network weights, or learning trajectory can help illuminate why and how your network has performed the way it did.

\section{Discussion}

%Summarize your goals, your approach, and your main findings, and state a conclusion indicating how well, overall, your goals have been met.

%Then, discuss shortcomings or limitations of your effort and indicate how these might be overcome in future work.

%%Also, indicate broader implications and potential future applications of the ideas and approach.

\section{Summary}

Citations:   We will not be sticklers for details of citation styles but do provide citations to literature you draw on in your paper, using the following format

\end{document}

